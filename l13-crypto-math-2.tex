\documentclass[11pt]{article}
\usepackage{fullpage}
\usepackage{parskip}    % do not indent paragraphs
\usepackage{amsfonts}
\usepackage{mathabx}    % \divides, \notdivides
%\usepackage{graphicx}
%\usepackage{subfigure}

\newcommand{\Zp}{\mathbb{Z}^{\ast}_p}
\newcommand{\Zn}{\mathbb{Z}^{\ast}_n}
\newcommand{\Qp}{\mathbb{Q}_n}
\newcommand{\Qn}{\mathbb{Q}_n}
\newcommand{\G}{\mathbb{G}}
\newcommand{\gen}[1]{\langle #1 \rangle}
%\newcommand{\divides}{\bigm|}
\newcommand{\definition}{\textbf{Definition:} }
\newcommand{\theoremof}[1]{\textbf{#1 Theorem:} }
\newcommand{\theorem}{\textbf{Theorem:} }
\newcommand{\note}{\emph{Note:} }
\newcommand{\example}{\emph{Example:} }

\begin{document}

\title{Crypto math II}
\author{Alin Tomescu\\
  \texttt{alinush@mit.edu}}

\maketitle

\begin{abstract}
A quick overview on group theory from Ron Rivest's 6.857 course in Spring 2015.
\end{abstract}

\section{Overview}

\begin{itemize}
  \item Group theory review
  \item Diffie-Hellman (DH) key exchane
  \item Five crypto groups:
  \begin{itemize}
    \item $\Zp$
    \item $\Qp$
    \item $\Zn$
    \item $\Qn$
    \item elliptic curves
  \end{itemize}
\end{itemize}

\section{Group theory review}

Here, we are talking about multiplicative groups (where the operation between
group elements is something \emph{resembling} multiplication)

\definition $(\G, \cdot)$ is a \emph{finite abelian group} of size $t$ if:

\begin{itemize}
  \item $\exists$ identity $1$ such that $\forall a \in \G, a\cdot 1 = 1\cdot a = a$
  \item $\forall a     \in \G, \exists b \in \G$ such that $a\cdot b = 1$
  \item $\forall a,b,c \in \G, a\cdot (b\cdot c) = (a\cdot b)\cdot c$
  \item $\forall a,b   \in \G, a\cdot b = b\cdot a$
\end{itemize}

\subsection{Order and generators}

\definition The \emph{order} of $a$ in $\G$ is denoted by $order(a)$ and is equal
to the least $u$ such that $a^u = 1$

\theoremof{Lagrange's} In a finite abelian group of size $t$, for all $a \in \G$,
$order(a) \divides t$

\theorem In a finite abelian group of size $t$, $\forall a \in \G, a^t = 1$

\example $a^{(p-1)} = 1, \forall a \in \Zp$ because $|\Zp| = 1$

\definition $\gen{a} = \{a^i : i \ge 0\} = $ subgroup generated by $a$.

\definition If $\gen{a} = \G$ then $\G$ is \emph{cyclic} and $a$ is a
\emph{generator} of $\G$.

\note $|\gen{a}| = order(a)$

\emph{Exercise:} In a finite abelian group $\G$ of order $t$, where $t$ is
prime, we have: $\forall a \in \G$, if $a \ne 1 \Rightarrow a$ is a generator of $\G$.

\emph{Solution:} We know that the size of any subgroup of $\G$ must divide $t$.
Since $t$ is prime, any subgroup can either have size $1$ or $t$. Thus, only
trivial subgroups can exist: the subgroup made up of the identity element
($\{1\}$) and $\G$ itself. Since $a \ne 1$, any subgroup generated by $a$ cannot
be equal to $\{1\}$ because it will have to contain $a$ itself which is
different than $1$.  Thus, if $a$ generates any subgroup, it has to generate
$\G$ itself. How do we know that $a$ generates any subgroup at all then? We know
$a \in G \Rightarrow a^u \in G, \forall u$ and, informally, we know that there
cannot be a $u, 1 < u < t$ such that $a^u = 1$ because that would create a
subgroup of $\G$ of size $u$, which would imply $u \divides t$, which would be
false since $t$ is prime.

\theorem $\Zp$ is always cyclic (i.e. there exists a generator within $\Zp$)

\subsection{Discrete logs}

\theorem If $\G$ is a cyclic group of order $t$ and generator $g$ then the relation
$x \leftrightarrow g^x$ is one-to-one between $[0, 1, \dots, t-1]$ and $\G$.

\end{document}
